\documentclass[a4paper,10pt,BCOR10mm,oneside,headsepline]{scrartcl}
\usepackage[ngerman]{babel}
\usepackage[utf8]{inputenc}
\usepackage{wasysym}% provides \ocircle and \Box
\usepackage{enumitem}% easy control of topsep and leftmargin for lists
\usepackage{color}% used for background color
\usepackage{forloop}% used for \Qrating and \Qlines
\usepackage{ifthen}% used for \Qitem and \QItem
\usepackage{typearea}
\usepackage{graphicx, wrapfig}
\usepackage{tabularx}
\usepackage[table]{xcolor}


\areaset{17cm}{26cm}
\setlength{\topmargin}{-1cm}
\usepackage{scrpage2}
\pagestyle{scrheadings}
\ihead{Fragebogen zum Nicht-HMD-Erlebnis}
\ohead{\pagemark}
\chead{}
\cfoot{}

%%%%%%%%%%%%%%%%%%%%%%%%%%%%%%%%%%%%%%%%%%%%%%%%%%%%%%%%%%%%
%% Beginning of questionnaire command definitions %%
%%%%%%%%%%%%%%%%%%%%%%%%%%%%%%%%%%%%%%%%%%%%%%%%%%%%%%%%%%%%
%%
%% 2010, 2012 by Sven Hartenstein
%% mail@svenhartenstein.de
%% http://www.svenhartenstein.de
%%
%% Please be warned that this is NOT a full-featured framework for
%% creating (all sorts of) questionnaires. Rather, it is a small
%% collection of LaTeX commands that I found useful when creating a
%% questionnaire. Feel free to copy and adjust any parts you like.
%% Most probably, you will want to change the commands, so that they
%% fit your taste.
%%
%% Also note that I am not a LaTeX expert! Things can very likely be
%% done much more elegant than I was able to. If you have suggestions
%% about what can be improved please send me an email. I intend to
%% add good tipps to my website and to name contributers of course.
%%
%% 10/2012: Thanks to karathan for the suggestion to put \noindent
%% before \rule!

%% \Qq = Questionaire question. Oh, this is just too simple. It helps
%% making it easy to globally change the appearance of questions.
\newcommand{\Qq}[1]{\textbf{#1}}

%% \QO = Circle or box to be ticked. Used both by direct call and by
%% \Qrating and \Qlist.
\newcommand{\QO}{$\Box$}% or: $\ocircle$

%% \Qrating = Automatically create a rating scale with NUM steps, like
%% this: 0--0--0--0--0.
\newcounter{qr}
\newcommand{\Qrating}[1]{\QO\forloop{qr}{1}{\value{qr} < #1}{---\QO}}

%% \Qline = Again, this is very simple. It helps setting the line
%% thickness globally. Used both by direct call and by \Qlines.
\newcommand{\Qline}[1]{\noindent\rule{#1}{0.6pt}}

%% \Qlines = Insert NUM lines with width=\linewith. You can change the
%% \vskip value to adjust the spacing.
\newcounter{ql}
\newcommand{\Qlines}[1]{\forloop{ql}{0}{\value{ql}<#1}{\vskip0.5em\Qline{\linewidth}}}

%% \Qlist = This is an environment very similar to itemize but with
%% \QO in front of each list item. Useful for classical multiple
%% choice. Change leftmargin and topsep accourding to your taste.
\newenvironment{Qlist}{%
	\renewcommand{\labelitemi}{\QO}
	\begin{itemize}[leftmargin=1.5em,topsep=-.5em]
	}{%
	\end{itemize}
}

%% \Qtab = A "tabulator simulation". The first argument is the
%% distance from the left margin. The second argument is content which
%% is indented within the current row.
\newlength{\qt}
\newcommand{\Qtab}[2]{
	\setlength{\qt}{\linewidth}
	\addtolength{\qt}{-#1}
	\hfill\parbox[t]{\qt}{\raggedright #2}
}

%% \Qitem = Item with automatic numbering. The first optional argument
%% can be used to create sub-items like 2a, 2b, 2c, ... The item
%% number is increased if the first argument is omitted or equals 'a'.
%% You will have to adjust this if you prefer a different numbering
%% scheme. Adjust topsep and leftmargin as needed.
\newcounter{itemnummer}
\newcommand{\Qitem}[2][]{% #1 optional, #2 notwendig
	\ifthenelse{\equal{#1}{}}{\stepcounter{itemnummer}}{}
	\ifthenelse{\equal{#1}{a}}{\stepcounter{itemnummer}}{}
	\begin{enumerate}[topsep=2pt,leftmargin=2.8em]
		\item[\textbf{\arabic{itemnummer}#1.}] #2
	\end{enumerate}
}

%% \QItem = Like \Qitem but with alternating background color. This
%% might be error prone as I hard-coded some lengths (-5.25pt and
%% -3pt)! I do not yet understand why I need them.
\definecolor{bgodd}{rgb}{0.8,0.8,0.8}
\definecolor{bgeven}{rgb}{0.9,0.9,0.9}
\newcounter{itemoddeven}
\newlength{\gb}
\newcommand{\QItem}[2][]{% #1 optional, #2 notwendig
	\setlength{\gb}{\linewidth}
	\addtolength{\gb}{-5.25pt}
	\ifthenelse{\equal{\value{itemoddeven}}{0}}{%
		\noindent\colorbox{bgeven}{\hskip-3pt\begin{minipage}{\gb}\Qitem[#1]{#2}\end{minipage}}%
		\stepcounter{itemoddeven}%
	}{%
		\noindent\colorbox{bgodd}{\hskip-3pt\begin{minipage}{\gb}\Qitem[#1]{#2}\end{minipage}}%
		\setcounter{itemoddeven}{0}%
	}
}

\renewcommand{\QO}{$\ocircle$}% Use circles now instead of boxes.

\newcommand{\gqq}[1]{\glqq{}#1\grqq{}}

%%%%%%%%%%%%%%%%%%%%%%%%%%%%%%%%%%%%%%%%%%%%%%%%%%%%%%%%%%%%
%% End of questionnaire command definitions %%
%%%%%%%%%%%%%%%%%%%%%%%%%%%%%%%%%%%%%%%%%%%%%%%%%%%%%%%%%%%%

\newcolumntype{P}{>{\centering\arraybackslash}p{0.75cm}}
\newcolumntype{L}{>{\raggedright\arraybackslash}m{0.2\textwidth}}
\newcolumntype{R}{>{\raggedleft\arraybackslash}m{0.2\textwidth}}

\newcommand{\printtblhdr}{%
	\hfill
	\begingroup
	\setlength\tabcolsep{0pt}%
	\begin{tabularx}{0.5\textwidth}{ @{} l *{3}X r @{} }
		\multicolumn{2}{l}{\bfseries\shortstack[l]{Komplett\\abgelehnt}}
		&&
		\multicolumn{2}{l}{\bfseries\shortstack[r]{Komplett\\zugestimmt}}
		\\
	\end{tabularx}
	\endgroup
}

\newcommand{\usetbl}{%
	\begin{tabular}{@{}*7{P}@{}}
		1 & 2 & 3 & 4 & 5 & 6 & 7 \\
	\end{tabular}
}

\newcommand{\usetblNew}{%
	\begin{tabular}{@{}|*7{P|}@{}}
		\hline
		1 & 2 & 3 & 4 & 5 & 6 & 7 \\
		\hline
	\end{tabular}
}

\definecolor{lighter-gray}{gray}{0.89}
\begin{document}
	
	\begin{center}
		\textbf{\huge Fragebogen zum Nicht-HMD-Erlebnis}
	\end{center}
	
	Teilnehmer: \Qline{5cm}
	\hfill
	Spiel: \Qline{5cm}
	
	\section{Fragen zur Benutzererfahrung}
	
	Bitte bewerten Sie die folgenden Beschreibungen Ihrer Einsch\"atzung nach an. Die Beschreibungen beziehen sich auf die Erfahrungen, die Sie in Ihrer Rolle erlebt haben und Ihr Spielerlebnis.
	
	\renewcommand{\arraystretch}{2}
	\rowcolors{1}{}{lighter-gray}
	\begin{center}
		\begin{tabular}{r c l}
			% https://measuringu.com/sus/
			\hline
			unerfreulich & \usetbl & erfreulich\\
			kreativ & \usetbl & phantasielos\\
			unverst\"andlich & \usetbl & verst\"andlich\\
			leicht zu lernen & \usetbl & schwer zu lernen\\
			langweilig & \usetbl & spanned\\
			schnell & \usetbl & langsam\\
			unberechenbar & \usetbl & voraussagbar\\
			originiell & \usetbl & konventionell\\
			uninteressant & \usetbl & interessant\\
			sicher & \usetbl & unsicher\\
			behindernt & \usetbl & unterst\"utzend\\
			kompliziert & \usetbl & einfach\\
			aktivierend & \usetbl & einschl\"afernd\\
			absto\ss{}end & \usetbl & anziehend\\
			\"ubersichtlich & \usetbl & verwirrend\\
			unangenehm & \usetbl & angenenehm\\
			aufger\"aumt & \usetbl & \"uberladen\\
			ineffizient & \usetbl & effizient\\
			attraktiv & \usetbl & unattraktiv\\
			unpragmatisch & \usetbl & pragmatisch\\
			konservativ & \usetbl & innovativ\\
			\hline
		\end{tabular} 
	\end{center}
	
	\section{Fragen zur Pr\"asenz}
	
	Bitte kreuzen Sie nachfolgend an, wie sehr Sie den jeweiligen Aussagen zustimmen oder sie ablehnen. M\"ochten Sie sich enthalten, kreuzen Sie bitte den Kreis neben der Zeile an.
	
	\rowcolors{2}{}{lighter-gray}
	\hspace{-1.5cm}
	\begin{tabular}{p{0.55\textwidth}  l | c}
		& \printtblhdr	\\ 
		\hline
		1. Ich konnte alle Ereignisse sehr gut kontrollieren. & \usetbl & O \\ % 1
		2. Die Interaktion mit der Umgebung war sehr nat\"urlich. & \usetbl & O \\ % 2
		3. Die Umgebung hat sofort auf meine Aktionen reagiert. & \usetbl & O \\ % 3
		4. Der Bewegungsmechanismus war sehr nat\"urlich. & \usetbl & O \\ % 4
		5. Ich konnte mich durch die graphische Darstellung stark in das Spiel hinein versetzen. & \usetbl & O \\ % 5
		6. Ich konnte alle Folgen meiner Handlungen vorausahnen. & \usetbl & O \\ % 7
		7. Alle Objekte im sichtbaren Spielbereich haben sich r\"aumlich sehr angenehm bewegt. & \usetbl & O \\ % 9
		8. Ich war komplett an der Erfahrung in der virtuellen Welt involviert. & \usetbl & O \\ % 11
		9. Meine M\"oglichkeiten Objekte oder den Spieler zu steuern waren sehr angenehm. & \usetbl & O \\ % 13
		10. Ich konnte mich sehr schnell an das Erlebnis der Spielwelt anpassen. & \usetbl & O \\ % 15
		11. Ich habe mein Zeitgef\"uhl bei der Bearbeitung der Aufgabe komplett verloren. & \usetbl & O \\ % 16
		12. Die Ger\"ate haben mich gar nicht bei der Ausf\"uhrung meiner Aufgabe gehindert. & \usetbl & O \\ % 17
		13. Ich konnte mich komplett auf die Ausf\"uhrung der Aufgabe konzentrieren. & \usetbl & O \\ % 18
		14. Ich habe mich mit der Fortbewegungs- und Interaktionsmethodik nach dem Experiment vertraut gef\"uhlt. & \usetbl & O \\ % 19
		\hline
	\end{tabular}
	\vspace{1cm} \\
	Anmerkungen:
	\Qlines{6}
	
	% 6. Ich war mir gar nicht mehr \"uber die reale Welt bewusst. & \usetbl \\ % 6
	% 8. Den Display und Controller habe ich so gar nicht wahr genommen. & \usetbl \\ % 8
	% 10. Die Bewegung in der virtuellen Welt hat mich komplett \"uberzeugt. & \usetbl \\ % 10
	% 12. Ich konnte mich sehr gut mit Ber\"uhrungsgesten umgucken/informieren. & \usetbl \\ % 12
	% 14. Ich habe keinerlei Verz\"ogerungen zwischen meinen Bewegungsaktionen und ihren erwarteten 		Folgen gesp\"urt. & \usetbl \\ % 14
	
	
\end{document}