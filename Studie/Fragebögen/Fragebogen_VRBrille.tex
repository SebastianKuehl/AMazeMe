\documentclass[a4paper,10pt,BCOR10mm,oneside,headsepline]{scrartcl}
\usepackage[ngerman]{babel}
\usepackage[utf8]{inputenc}
\usepackage{wasysym}% provides \ocircle and \Box
\usepackage{enumitem}% easy control of topsep and leftmargin for lists
\usepackage{color}% used for background color
\usepackage{forloop}% used for \Qrating and \Qlines
\usepackage{ifthen}% used for \Qitem and \QItem
\usepackage{typearea}
\usepackage{graphicx, wrapfig}
\usepackage{tabularx}
\usepackage[table]{xcolor}


\areaset{17cm}{26cm}
\setlength{\topmargin}{-1cm}
\usepackage{scrpage2}
\pagestyle{scrheadings}
\ihead{Fragebogen zur VR-Brille}
\ohead{\pagemark}
\chead{}
\cfoot{}

%%%%%%%%%%%%%%%%%%%%%%%%%%%%%%%%%%%%%%%%%%%%%%%%%%%%%%%%%%%%
%% Beginning of questionnaire command definitions %%
%%%%%%%%%%%%%%%%%%%%%%%%%%%%%%%%%%%%%%%%%%%%%%%%%%%%%%%%%%%%
%%
%% 2010, 2012 by Sven Hartenstein
%% mail@svenhartenstein.de
%% http://www.svenhartenstein.de
%%
%% Please be warned that this is NOT a full-featured framework for
%% creating (all sorts of) questionnaires. Rather, it is a small
%% collection of LaTeX commands that I found useful when creating a
%% questionnaire. Feel free to copy and adjust any parts you like.
%% Most probably, you will want to change the commands, so that they
%% fit your taste.
%%
%% Also note that I am not a LaTeX expert! Things can very likely be
%% done much more elegant than I was able to. If you have suggestions
%% about what can be improved please send me an email. I intend to
%% add good tipps to my website and to name contributers of course.
%%
%% 10/2012: Thanks to karathan for the suggestion to put \noindent
%% before \rule!

%% \Qq = Questionaire question. Oh, this is just too simple. It helps
%% making it easy to globally change the appearance of questions.
\newcommand{\Qq}[1]{\textbf{#1}}

%% \QO = Circle or box to be ticked. Used both by direct call and by
%% \Qrating and \Qlist.
\newcommand{\QO}{$\Box$}% or: $\ocircle$

%% \Qrating = Automatically create a rating scale with NUM steps, like
%% this: 0--0--0--0--0.
\newcounter{qr}
\newcommand{\Qrating}[1]{\QO\forloop{qr}{1}{\value{qr} < #1}{---\QO}}

%% \Qline = Again, this is very simple. It helps setting the line
%% thickness globally. Used both by direct call and by \Qlines.
\newcommand{\Qline}[1]{\noindent\rule{#1}{0.6pt}}

%% \Qlines = Insert NUM lines with width=\linewith. You can change the
%% \vskip value to adjust the spacing.
\newcounter{ql}
\newcommand{\Qlines}[1]{\forloop{ql}{0}{\value{ql}<#1}{\vskip0.5em\Qline{\linewidth}}}

%% \Qlist = This is an environment very similar to itemize but with
%% \QO in front of each list item. Useful for classical multiple
%% choice. Change leftmargin and topsep accourding to your taste.
\newenvironment{Qlist}{%
	\renewcommand{\labelitemi}{\QO}
	\begin{itemize}[leftmargin=1.5em,topsep=-.5em]
	}{%
	\end{itemize}
}

%% \Qtab = A "tabulator simulation". The first argument is the
%% distance from the left margin. The second argument is content which
%% is indented within the current row.
\newlength{\qt}
\newcommand{\Qtab}[2]{
	\setlength{\qt}{\linewidth}
	\addtolength{\qt}{-#1}
	\hfill\parbox[t]{\qt}{\raggedright #2}
}

%% \Qitem = Item with automatic numbering. The first optional argument
%% can be used to create sub-items like 2a, 2b, 2c, ... The item
%% number is increased if the first argument is omitted or equals 'a'.
%% You will have to adjust this if you prefer a different numbering
%% scheme. Adjust topsep and leftmargin as needed.
\newcounter{itemnummer}
\newcommand{\Qitem}[2][]{% #1 optional, #2 notwendig
	\ifthenelse{\equal{#1}{}}{\stepcounter{itemnummer}}{}
	\ifthenelse{\equal{#1}{a}}{\stepcounter{itemnummer}}{}
	\begin{enumerate}[topsep=2pt,leftmargin=2.8em]
		\item[\textbf{\arabic{itemnummer}#1.}] #2
	\end{enumerate}
}

%% \QItem = Like \Qitem but with alternating background color. This
%% might be error prone as I hard-coded some lengths (-5.25pt and
%% -3pt)! I do not yet understand why I need them.
\definecolor{bgodd}{rgb}{0.8,0.8,0.8}
\definecolor{bgeven}{rgb}{0.9,0.9,0.9}
\newcounter{itemoddeven}
\newlength{\gb}
\newcommand{\QItem}[2][]{% #1 optional, #2 notwendig
	\setlength{\gb}{\linewidth}
	\addtolength{\gb}{-5.25pt}
	\ifthenelse{\equal{\value{itemoddeven}}{0}}{%
		\noindent\colorbox{bgeven}{\hskip-3pt\begin{minipage}{\gb}\Qitem[#1]{#2}\end{minipage}}%
		\stepcounter{itemoddeven}%
	}{%
		\noindent\colorbox{bgodd}{\hskip-3pt\begin{minipage}{\gb}\Qitem[#1]{#2}\end{minipage}}%
		\setcounter{itemoddeven}{0}%
	}
}

\renewcommand{\QO}{$\ocircle$}% Use circles now instead of boxes.

\newcommand{\gqq}[1]{\glqq{}#1\grqq{}}

%%%%%%%%%%%%%%%%%%%%%%%%%%%%%%%%%%%%%%%%%%%%%%%%%%%%%%%%%%%%
%% End of questionnaire command definitions %%
%%%%%%%%%%%%%%%%%%%%%%%%%%%%%%%%%%%%%%%%%%%%%%%%%%%%%%%%%%%%

\newcolumntype{P}{>{\centering\arraybackslash}p{0.75cm}}
\newcolumntype{L}{>{\raggedright\arraybackslash}m{0.2\textwidth}}
\newcolumntype{R}{>{\raggedleft\arraybackslash}m{0.2\textwidth}}

\newcommand{\printtblhdr}{%
	\hfill
	\begingroup
	\setlength\tabcolsep{0pt}%
	\begin{tabularx}{0.4\textwidth}{ @{} l *{3}X r @{} }
		\multicolumn{2}{l}{\bfseries\shortstack[l]{Komplett\\abgelehnt}}
		&&
		\multicolumn{2}{l}{\bfseries\shortstack[r]{Komplett\\zugestimmt}}
		\\
	\end{tabularx}
	\endgroup
}

\newcommand{\usetbl}{%
	\begin{tabular}{@{}|*5{P|}@{}}
		\hline
		1 & 2 & 3 & 4 & 5 \\
		\hline
	\end{tabular}
}

\definecolor{lighter-gray}{gray}{0.89}
\begin{document}
	
	\begin{center}
		\textbf{\huge Fragebogen zum VR-Brillen-gesteuerten Viewport}
	\end{center}
	
	Teilnehmer: \Qline{5cm}
	\section{System Usability}
	
	\printtblhdr
	
	\renewcommand{\arraystretch}{2}
	\rowcolors{1}{}{lighter-gray}
	\begin{tabular}{p{0.575\textwidth}  l}
		% https://measuringu.com/sus/
		1. Ich denke ich w\"urde das System \"ofters benutzen wollen. & \usetbl \\
		2. Ich fand das System unn\"otig komplex. & \usetbl \\
		3. Ich fand das System einfach zu benutzen. & \usetbl \\
		4. Ich glaube ich br\"auchte Hilfe/Unterst\"utzung von einer technisch erfahreneren Person, um das System benutzen zu k\"onnen. & \usetbl \\
		5. Ich fand die verschiedenen Funktionalit\"aten des Systems waren gut integriert. & \usetbl \\
		6. Ich dachte es gab zu viel Inkonsistenz in dem System. & \usetbl \\
		7. Ich k\"onnte mir vorstellen, dass viele Personen die Benutzung des Systems schnell erlernen w\"urden. & \usetbl \\
		8. Ich fand das System in der Benutzung sehr schwerf\"allig. & \usetbl \\
		9. Ich war bei der Benutzung des Systems sehr selbstbewusst. & \usetbl \\
		10. Ich musste viele Dinge lernen, bevor ich mit dem System arbeiten konnte. & \usetbl \\		
	\end{tabular} 
	
	Weitere Anmerkungen:
	\Qlines{4}
	
	\section{Steuerung}
	
	\rowcolors{1}{}{white}
	\printtblhdr	
	\rowcolors{1}{}{lighter-gray}
	\begin{tabular}{p{0.575\textwidth}  l}
		% https://measuringu.com/sus/
		1. Ich fand die Steuerung des Systems schnell erlernbar. & \usetbl \\
		2. Ich fand die Steuerung des Systems angenehm. & \usetbl \\
		3. Ich finde die Steuerung des Systems k\"onnte angepasst werden. & \usetbl \\
		4. Ich finde die Steuerung des Systems mit einem realen Fernglas vergleichbar. & \usetbl \\
		5. Ich wusste sofort, wie ich das System steuern kann. & \usetbl \\
		6. Meine Eingabe bei der Steuerung des Systems hat genau den Effekt erzielt, den ich erwartet habe. & \usetbl \\
		7. Ich f\"uhlte mich bei der Steuerung des Systems in der Bewegung eingeschr\"ankt. & \usetbl \\
		8. Ich f\"uhlte mich bei der Steuerung des Systems selbstbewusst. & \usetbl \\
		9. Ich konnte mich w\"ahrend der Benutzung des Systems frei Bewegen. & \usetbl \\
		10. Nach einer Pause bei der Benutzung des Systems musste ich Korrekturen des Bildausschnitts durch erneutes Steuern des Systems vornehmen. & \usetbl \\
	\end{tabular}
	
	Weitere Anmerkungen zur Steuerung:
	\Qlines{4}
	
	\section{Graphische Darstellung/Inhalt/Anzeige}
	
	\rowcolors{1}{}{white}
	\printtblhdr	
	\rowcolors{1}{}{lighter-gray}
	\begin{tabular}{p{0.575\textwidth}  l}
		% https://measuringu.com/sus/
		1. Ich fand die Aufl\"osung des Systems hoch genug. & \usetbl \\
		2. Der Bildschirm des Systems k\"onnte gr\"o\ss{}er sein. & \usetbl \\
		3. Ich fand den Sichtbereich des Systems gro\ss{} genug. & \usetbl \\
		4. Ich konnte Objekte in der Darstellung mit dem System schnell erkennen. & \usetbl \\
		5. Ich konnte Details mit dem System genau erkennen. & \usetbl \\
		6. Ich fand das angezeigte Bild des Systems mit dem eines Fernglases vergleichbar. & \usetbl \\
		7. Ich fand die Zoomst\"arke des Systems mit der eines Fernglases vergleichbar. & \usetbl \\
		8. Graphische Effekte wie Vignette w\"urden das System realistischer machen. & \usetbl \\
	\end{tabular} \par
	
	Weitere Anmerkungen zur graphischen Darstellung/Inhalt/Anzeige:
	\Qlines{4}
	
	\section{Gebrauchstauglichkeit}
	
	\rowcolors{1}{}{white}
	\printtblhdr	
	\rowcolors{1}{}{lighter-gray}
	\begin{tabular}{p{0.575\textwidth}  l}
		% https://measuringu.com/sus/
		1. Ich konnte Objekte mit dem System schnell finden. & \usetbl \\
		2. Ich musste mich bei der Benutzung des Systems konzentrieren. & \usetbl \\
		3. Ich hatte Schwierigkeiten bei der Benutzung des Systems. & \usetbl \\
		4. Ich fand die Handhabung des Systems mit der eines Fernglases vergleichbar. & \usetbl \\
		5. Ich k\"onnte das System \"uber l\"angere Zeit benutzen. & \usetbl \\
		6. Ich konnte das System nach Nicht-Benutzung schnell aufgreifen und benutzen. & \usetbl \\
		7. Ich hatte Schwierigkeiten das System nach erfolgreicher Benutzung ab-/wegzulegen. & \usetbl \\
	\end{tabular} \par
	
	\newpage 
	
	Weitere Anmerkungen zur Gebrauchstauglichkeit:
	\Qlines{4}
	
	\section{R\"aumliche Wahrnehmung}
	
	\rowcolors{1}{}{white}
	\printtblhdr	
	\rowcolors{1}{}{lighter-gray}
	\begin{tabular}{p{0.575\textwidth}  l}
		% https://measuringu.com/sus/
		1. Bei der Benutzung des Systems hatte ich ein Gef\"uhl der r\"aumlichen Wahrnehmung. & \usetbl \\
		2. Ich finde das System erzeugt den Effekt von Tiefe und Raum. & \usetbl \\
		3. Ich finde der 3D Effekt des Systems k\"onnte verbessert werden. & \usetbl \\
		4. Ich k\"onnte das System auch \"uber l\"angere Zeit betrachten. & \usetbl \\
		5. Ich habe meine reale Umgebung bei der Benutzung des Systems nicht wahrgenommen. & \usetbl \\
		6. Ich habe die Darstellung des Systems als Bild und nicht als r\"aumliche Darstellung wahrgenommen. & \usetbl \\
		7. Ich hatte bei dem System das Gef\"uhl, dass die virtuelle Welt hinter mir weiter ginge. & \usetbl \\
		8. Ich hatte bei dem System das Gef\"uhl die gezeigten Gegenst\"ande greifen zu k\"onnen. & \usetbl \\
		9. Ich hatte bei dem System das Gef\"uhl im virtuellen Raum anwesend zu sein. & \usetbl \\
	\end{tabular} \par
	
	Weitere Anmerkungen zur r\"aumlichen Wahrnehmung:
	\Qlines{4}
	
\end{document}